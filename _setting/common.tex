% Common settings for various templates

% allows using mathematical expression
\usepackage{amsmath,amssymb,amsfonts,amsthm,mathtools}
\everymath{\displaystyle}
\DeclareMathOperator*{\argmax}{arg\,max}
\DeclareMathOperator*{\argmin}{arg\,min}

% allows including images
\usepackage{graphicx}

% advanced citation option
\usepackage{natbib}

% font settings
\usepackage{fontspec} % pdflatex is NOT allowed!
\defaultfontfeatures[\rmfamily,\sffamily]{Ligatures=TeX}
\setsansfont[
    Path=_fonts/TeX Gyre/,
    Scale=MatchUppercase,
    UprightFont={*-regular.otf},
    ItalicFont={*-italic.otf},
    BoldFont={*-bold.otf},
    BoldItalicFont={*-bolditalic.otf}
]{texgyreheros}
\setmonofont[
    Path=_fonts/DejaVu/,
    Scale=MatchLowercase,
    UprightFont={*.ttf},
    ItalicFont={*-Oblique.ttf},
    BoldFont={*-Bold.ttf},
    BoldItalicFont={*-BoldOblique.ttf}
]{DejaVuSansMono}
% \renewcommand{\familydefault}{\sfdefault} % all text (except math) in sans serif
% \usepackage[helvet]{sfmath}\usepackage{eulervm} % math in sans serif

% allows using korean text
%\usepackage[hangul]{kotex}
\usepackage{kotex}
\setmainhangulfont[
    Path=_fonts/KoPub/,
    UprightFont={* Light.otf},
    ItalicFont={* Light.otf}, ItalicFeatures={FakeSlant=.167},
    BoldFont={* Bold.otf},
    BoldItalicFont={* Bold.otf}, BoldItalicFeatures={FakeSlant=.167}
]{KoPubBatang_Pro}
\setsanshangulfont[
    Path=_fonts/KoPub/,
    Scale=MatchUppercase,
    UprightFont={* Medium.otf},
    ItalicFont={* Medium.otf},ItalicFeatures={FakeSlant=.167},
    BoldFont={* Bold.otf},
    BoldItalicFont={* Bold.otf}, BoldItalicFeatures={FakeSlant=.167}
]{KoPubDotum_Pro}
\setmonohangulfont[
    Path=_fonts/KoPub/,
    Scale=MatchLowercase,
    UprightFont={* Medium.otf},
    ItalicFont={* Medium.otf},ItalicFeatures={FakeSlant=.167},
    BoldFont={* Bold.otf},
    BoldItalicFont={* Bold.otf}, BoldItalicFeatures={FakeSlant=.167}
]{KoPubDotum_Pro}

% various packages for table
\usepackage{booktabs,array,multirow}
% allows customizing list environments
\usepackage[inline]{enumitem}
\setlist[enumerate,1]{label=\arabic*.}
\setlist[enumerate,2]{label=(\alph*)}
\setlist[enumerate,3]{label=\roman*.}
\setlist[enumerate,4]{label=\Alph*.}
\setlist[itemize,1]{label=$\triangleright$}
\setlist[itemize,2]{label=-}
\setlist[itemize,3]{label=$\ast$}
\setlist[itemize,4]{label=$\cdot$}

% define color scheme
\usepackage{xcolor}
% Bootstrap-like color scheme
\definecolor{black}{HTML}{000000}
\definecolor{gray-darker}{HTML}{222222}
\definecolor{gray-dark}{HTML}{333333}
\definecolor{gray}{HTML}{555555}
\definecolor{gray-light}{HTML}{777777}
\definecolor{gray-lighter}{HTML}{EEEEEE}
\definecolor{white}{HTML}{FFFFFF}

\definecolor{primary}{HTML}{428BCA}

\definecolor{success}{HTML}{5CB85C}
\definecolor{success-bg}{HTML}{DFF0D8}
\definecolor{success-text}{HTML}{3C763D}

\definecolor{info}{HTML}{5BC0DE}
\definecolor{info-bg}{HTML}{D9EDF7}
\definecolor{info-text}{HTML}{31708F}

\definecolor{warning}{HTML}{F0AD4E}
\definecolor{warning-bg}{HTML}{FCF8E3}
\definecolor{warning-text}{HTML}{8A6D3B}

\definecolor{danger}{HTML}{D9534F}
\definecolor{danger-bg}{HTML}{F2DEDE}
\definecolor{danger-text}{HTML}{A94442}

\colorlet{background}{white}
\colorlet{text}{gray-dark}

\definecolor{listing-bg}{HTML}{F5F5F5}
\definecolor{listing-border}{HTML}{CCCCCC}
\definecolor{listing-kwd}{HTML}{2F6F9F}
\definecolor{listing-str}{HTML}{D44950}

\definecolor{code-bg}{HTML}{F9F2F4}
\definecolor{code-text}{HTML}{C7254E}
\colorlet{kbd-bg}{gray}
\colorlet{kbd-text}{gray-lighter}
% set document color
\makeatletter
\newcommand{\globalcolor}[1]{%
    \color{#1}\global\let\default@color\current@color
}
\makeatother
\AtBeginDocument{\globalcolor{text}\pagecolor{background}} % change color here

% graphic package : tikz
\usepackage{tikz}

% allows include programming codes
\usepackage{listings}
\lstset {
    tabsize=4,
    numbers=left,
    firstnumber=auto,
    numberstyle=\color{listing-border}\tiny\ttfamily,
    numbersep=8pt,
    stepnumber=1,
    frame=single,
    frameround=tttt,
    rulesep=8pt,
    breaklines=true,
    captionpos=b,
    showspaces=false,
    keepspaces=true,
    showstringspaces=false,
    showtabs=false,
    basicstyle=\small\ttfamily,
    backgroundcolor=\color{listing-bg},
    rulecolor=\color{listing-border},
    commentstyle=\color{gray-light},
    stringstyle=\color{listing-str},
    keywordstyle=\color{listing-kwd},
    upquote=true,
}
% inline environment: code / keyboard
\lstdefinestyle{code}{
    basicstyle=\small\ttfamily\color{code-text},
    keywordstyle={}, commentstyle={}, stringstyle={},
}
\newcommand\code[2][]{
    \tikz[overlay]\node[
        fill=code-bg, inner sep=2pt,
        text depth=0.2ex, text height=1.25ex,
        anchor=text, rectangle, rounded corners=1mm]
        {\lstinline[style=code]§#2§};
    \phantom{\lstinline§#2§\hskip-.8ex}
}
\lstdefinestyle{kbd}{
    basicstyle=\small\ttfamily\color{kbd-text},
    keywordstyle={}, commentstyle={}, stringstyle={},
}
\newcommand\kbd[2][]{
    \tikz[overlay]\node[
        fill=kbd-bg, inner sep=2pt,
        text depth=0.2ex, text height=1.25ex,
        anchor=text, rectangle, rounded corners=1mm]
        {\lstinline[style=kbd]§#2§};
    \phantom{\lstinline§#2§\hskip-.8ex}
}

% allows using algorithm environment
\usepackage{algorithm}
\usepackage[noend]{algpseudocode}
\algnewcommand\algorithmicto{\textbf{to}}
\algnewcommand\algorithmicdownto{\textbf{downto}}
\algdef{SE}[FOR]{ForTo}{EndForTo}[2]{\algorithmicfor\ #1 \algorithmicto\ #2 \algorithmicdo}{\algorithmicend\ \algorithmicfor}
\algdef{SnE}[FOR]{ForTo}{EndForTo}[2]{\algorithmicfor\ #1 \algorithmicto\ #2 \algorithmicdo}
\algdef{SE}[FOR]{ForDownTo}{EndForDownTo}[2]{\algorithmicfor\ #1 \algorithmicdownto\ #2 \algorithmicdo}{\algorithmicend\ \algorithmicfor}
\algdef{SnE}[FOR]{ForDownTo}{EndForDownTo}[2]{\algorithmicfor\ #1 \algorithmicdownto\ #2 \algorithmicdo}
