% Common settings for various templates

% allows using mathematical expression
\usepackage{amsmath,amssymb,amsfonts,amsthm,mathtools}
\everymath{\displaystyle}
\DeclareMathOperator*{\argmax}{arg\,max}
\DeclareMathOperator*{\argmin}{arg\,min}

% allows including images
\usepackage{graphicx}

% advanced citation option
\usepackage{natbib}

% font settings
\usepackage{fontspec} % pdflatex is NOT allowed!
\defaultfontfeatures[\rmfamily,\sffamily]{Ligatures=TeX}
\setsansfont[
    Path=_fonts/TeX Gyre/,
    Scale=MatchUppercase,
    UprightFont={*-regular.otf},
    ItalicFont={*-italic.otf},
    BoldFont={*-bold.otf},
    BoldItalicFont={*-bolditalic.otf}
]{texgyreheros}
\setmonofont[
    Path=_fonts/DejaVu/,
    Scale=MatchLowercase,
    UprightFont={*.ttf},
    ItalicFont={*-Oblique.ttf},
    BoldFont={*-Bold.ttf},
    BoldItalicFont={*-BoldOblique.ttf}
]{DejaVuSansMono}
% \renewcommand{\familydefault}{\sfdefault} % all text (except math) in sans serif
% \usepackage[helvet]{eulervm,sfmath} % math in sans serif

% allows using korean text
%\usepackage[hangul]{kotex}
\usepackage{kotex}
\setmainhangulfont[
    Path=_fonts/KoPub/,
    UprightFont={* Light.otf},
    ItalicFont={* Light.otf}, ItalicFeatures={FakeSlant=.167},
    BoldFont={* Bold.otf},
    BoldItalicFont={* Bold.otf}, BoldItalicFeatures={FakeSlant=.167}
]{KoPubBatang_Pro}
\setsanshangulfont[
    Path=_fonts/KoPub/,
    Scale=MatchUppercase,
    UprightFont={* Medium.otf},
    ItalicFont={* Medium.otf},ItalicFeatures={FakeSlant=.167},
    BoldFont={* Bold.otf},
    BoldItalicFont={* Bold.otf}, BoldItalicFeatures={FakeSlant=.167}
]{KoPubDotum_Pro}
\setmonohangulfont[
    Path=_fonts/KoPub/,
    Scale=MatchLowercase,
    UprightFont={* Medium.otf},
    ItalicFont={* Medium.otf},ItalicFeatures={FakeSlant=.167},
    BoldFont={* Bold.otf},
    BoldItalicFont={* Bold.otf}, BoldItalicFeatures={FakeSlant=.167}
]{KoPubDotum_Pro}

% various packages for table
\usepackage{booktabs,array,multirow}
% allows customizing list environments
\usepackage[inline]{enumitem}
% \renewcommand{\labelitemi}{}
% \renewcommand{\labelitemii}{}
% \renewcommand{\labelitemiii}{}
% \renewcommand{\labelitemiv}{}

% define color scheme
\usepackage{xcolor}
% Bootstrap-like color scheme
\definecolor{black}{HTML}{000000}
\definecolor{gray-darker}{HTML}{222222}
\definecolor{gray-dark}{HTML}{333333}
\definecolor{gray}{HTML}{555555}
\definecolor{gray-light}{HTML}{777777}
\definecolor{gray-lighter}{HTML}{EEEEEE}
\definecolor{white}{HTML}{FFFFFF}
\definecolor{primary}{HTML}{428BCA}
\definecolor{primary-darken}{HTML}{357EBD}
\definecolor{success}{HTML}{5CB85C}
\definecolor{success-darken}{HTML}{4FAB4F}
\definecolor{info}{HTML}{5BC0DE}
\definecolor{info-darken}{HTML}{4EB3D1}
\definecolor{warning}{HTML}{F0AD4E}
\definecolor{warning-darken}{HTML}{E3A041}
\definecolor{danger}{HTML}{D9534F}
\definecolor{danger-darken}{HTML}{CC4642}
% GitHub source code style color scheme
\definecolor{githubred}{HTML}{DD1144}
\definecolor{githubgreen}{HTML}{008080}
\definecolor{githubblue}{HTML}{000080}
\definecolor{githubgrey}{HTML}{999988}
% set document color
\makeatletter
\newcommand{\globalcolor}[1]{%
    \color{#1}\global\let\default@color\current@color
}
\makeatother
\AtBeginDocument{\globalcolor{gray-dark}\pagecolor{white}} % change color here

% allows include programming codes
\usepackage{listings}
\lstset {
    tabsize=4,
    %numbers=left,
    firstnumber=auto,
    numberstyle=\color{githubgrey}\tiny,
    numbersep=8pt,
    stepnumber=1,
    %frame=single,
    frameround=tttt,
    rulesep=8pt,
    breaklines=true,
    captionpos=b,
    showspaces=false,
    keepspaces=true,
    showstringspaces=false,
    showtabs=false,
    basicstyle=\small\ttfamily,
    commentstyle=\color{githubgreen},
    stringstyle=\color{githubred},
    keywordstyle=\color{githubblue},
    upquote=true,
}

% allows using algorithm environment
\usepackage{algorithm}
\usepackage[noend]{algpseudocode}
\algnewcommand\algorithmicto{\textbf{to}}
\algnewcommand\algorithmicdownto{\textbf{downto}}
\algdef{SE}[FOR]{ForTo}{EndForTo}[2]{\algorithmicfor\ #1 \algorithmicto\ #2 \algorithmicdo}{\algorithmicend\ \algorithmicfor}
\algdef{SnE}[FOR]{ForTo}{EndForTo}[2]{\algorithmicfor\ #1 \algorithmicto\ #2 \algorithmicdo}
\algdef{SE}[FOR]{ForDownTo}{EndForDownTo}[2]{\algorithmicfor\ #1 \algorithmicdownto\ #2 \algorithmicdo}{\algorithmicend\ \algorithmicfor}
\algdef{SnE}[FOR]{ForDownTo}{EndForDownTo}[2]{\algorithmicfor\ #1 \algorithmicdownto\ #2 \algorithmicdo}

% graphic package : tikz
\usepackage{tikz}
\usetikzlibrary{arrows,automata,backgrounds,calc,calendar,chains,er,intersections,matrix,mindmap,folding,patterns,petri,plothandlers,plotmarks,shadows,shapes.geometric,shapes.misc,spy,topaths,trees}
